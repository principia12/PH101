\documentclass[titlepage]{article}

\usepackage{lipsum} % Package to generate dummy text throughout this template

\usepackage[sc]{mathpazo} % Use the Palatino font
\usepackage[T1]{fontenc} % Use 8-bit encoding that has 256 glyphs
\linespread{1.05} % Line spacing - Palatino needs more space between lines
\usepackage{microtype} % Slightly tweak font spacing for aesthetics

\usepackage[hmarginratio=1:1,top=32mm,columnsep=20pt]{geometry} % Document margins
\usepackage{multicol} % Used for the two-column layout of the document
\usepackage[hang, small,labelfont=bf,up,textfont=it,up]{caption} % Custom captions under/above floats in tables or figures
\usepackage{booktabs} % Horizontal rules in tables
\usepackage{float} % Required for tables and figures in the multi-column environment - they need to be placed in specific locations with the [H] (e.g. \begin{table}[H])
\usepackage{hyperref} % For hyperlinks in the PDF

\usepackage{lettrine} % The lettrine is the first enlarged letter at the beginning of the text
\usepackage{paralist} % Used for the compactitem environment which makes bullet points with less space between them

\usepackage{braket}
\usepackage{array}
\usepackage{calc}
\usepackage{graphicx}
\usepackage{listings}
\usepackage{color}
\usepackage[table,xcdraw]{xcolor}
\usepackage{adjustbox}
\usepackage{kotex}

\graphicspath{{images/}}

\definecolor{dkgreen}{rgb}{0,0.6,0}
\definecolor{gray}{rgb}{0.5,0.5,0.5}
\definecolor{mauve}{rgb}{0.58,0,0.82}



\hypersetup{%
    pdfborder = {0 0 0}
}



\usepackage{abstract} % Allows abstract customization
\renewcommand{\abstractnamefont}{\normalfont\bfseries} % Set the "Abstract" text to bold
\renewcommand{\abstracttextfont}{\normalfont\small\itshape} % Set the abstract itself to small italic text

\usepackage{titlesec} % Allows customization of titles
\renewcommand\thesection{\Roman{section}} % Roman numerals for the sections
\renewcommand\thesubsection{\Roman{subsection}} % Roman numerals for subsections
\titleformat{\section}[block]{\large\scshape\centering}{\thesection.}{1em}{} % Change the look of the section titles
\titleformat{\subsection}[block]{\large}{\thesubsection.}{1em}{} % Change the look of the section titles

\usepackage{fancyhdr} % Headers and footers
\pagestyle{fancy} % All pages have headers and footers
\fancyhead{} % Blank out the default header
\fancyfoot{} % Blank out the default footer
\fancyhead[C]{ PH101 Introductory Physics } % Custom header text
\fancyfoot[RO,LE]{\thepage} % Custom footer text

%----------------------------------------------------------------------------------------
%	TITLE SECTION
%----------------------------------------------------------------------------------------

\title{\vspace{-15mm}\fontsize{24pt}{10pt}\selectfont\textbf{Hitchhiker's Guide to the\\ Introductory Physics}} % Article title

\author{
\large
\textsc{Seungwoo Schin}\\[2mm]
\normalsize School of Computing, KAIST \\ % Your institution
\vspace{-5mm}
}
\date{}

%----------------------------------------------------------------------------------------

\begin{document}

%\maketitle % Insert title

%\thispagestyle{fancy} % All pages have headers and footers

\begin{titlepage}
\maketitle % Insert title

\end{titlepage}


%----------------------------------------------------------------------------------------
%	ARTICLE CONTENTS
%----------------------------------------------------------------------------------------

\thispagestyle{fancy} % All pages have headers and footers

\tableofcontents

\newpage


%----------------------------------------------------------------------------------------
%	ARTICLE CONTENTS
%----------------------------------------------------------------------------------------


\lstset{frame=tb,
  language=Python,
  aboveskip=3mm,
  belowskip=3mm,
  showstringspaces=false,
  columns=flexible,
  basicstyle={\small\ttfamily},
  numbers=none,
  numberstyle=\tiny\color{gray},
  keywordstyle=\color{blue},
  commentstyle=\color{dkgreen},
  stringstyle=\color{mauve},
  breaklines=true,
  breakatwhitespace=true
  tabsize=3
}
\section*{Preface \& General Plan} 



\newpage
\section{Short Recap on Mathematical Tools}

\subsection{Calculus}

\subsubsection{Coordinates}
\label{chap:coordinates}


\paragraph{Various Coordinates} 
좌표계란 공간 내의 임의의 한 점을 어떤 수, 또는 어떤 수들의 순서쌍으로 나타내는 함수이다. 예를 들어서, gps에서 사용하는 위도/경도 등은 지표면상의 위치를 나타내는 하나의 좌표계이다. 본 수업에서 다룰 좌표계에는 크게 4가지의 좌표계가 있다. 

\begin{itemize}
\item{직교좌표계(Cartesian Coordinate)} $(x,y,z)$
\item{극좌표계(Polar Coordinate)} $(r, \theta)$ 점대칭인 2차원 상황에서 많이 이용된다. 
\item{원통 좌표계(Cylindrical Coordinate)} $(r,\theta,z)$원기둥 표면 위의 한 점을 나타내기에 적합하며, 축대칭이면서 동시에 극좌표계를 사용하는 것이 적합할 때 이용된다. 
\item{구면좌표계(Spherical Coordinate)} $(\rho,\theta,\phi)$구면상의 한 점을 나타내기에 적합하다. 보통 점대칭인 상황을 기술할 때 사용하기 용이하다. 물리학에서의 많은 상황은 이러한 점대칭 상황을 가정한다. \footnote{https://youtu.be/StHMKdvuHN0}
\end{itemize}

\paragraph{Transformation between Coordinate Systems} 

\begin{itemize}
\item Cartesian to Polar Coordinate 
\begin{equation}
        r =  \sqrt{x^2+y^2} 
\end{equation}
\begin{equation}
        \theta = tan^{-1} y/x
\end{equation}
\item Cartesian to Cylindrical Coordinate
\item Cartesian to Spherical Coordinate
\end{itemize}



\subsubsection{Vector} 

\paragraph{Definition} 벡터의 정의는 쓰고자 하는 범위에 따라서 여러가지로 달라지나, 여기서는 가장 naive한 정의인 크기와 방향을 가지는 객체라고 정의\footnote{벡터의 정의는 다양하며, 여기서 말하는 크기와 방향이라는 개념 또한 수학적으로 애매하지만 일단은 넘어가고 차후에 더 수학적으로 엄밀한 정의를 할 수 있을 것이다.} 하자. 이와 대응되는 개념의 스칼라는 크기만 가진 개념으로 정의하자. 

예를 들어서 온도나 길이 등은 크기만 가지는 개념으로, 스칼라양으로 볼 수 있다. 이와는 다르게, 힘 같은 것은 크기와 방향을 동시에 가지므로 벡터로 생각할 수 있다. 

\paragraph{Operations} 

\begin{itemize}
\item Addition, Subtraction 
\item Scalar Multiplication 
\item Inner Product
\item Outer Product (3차원에서만 적용가능)
\end{itemize}


\subsubsection{Multivariable Functions} 

\paragraph{Recap on Definition of Function}  함수는 다음과 같은 3-tuple로 정의된다. 
\begin{definition} 
함수 f는 (X, Y, f)로 정의되며, 
\begin{itemize} 
\item X (range:set) 정의역이라고도 하며, 함수가 받는 입력의 집합이다. 
\item Y (range:set) 공역이라고 하며, 함수가 반환하는 출력의 집합이다. 
\item $f \subset X \times Y$ (range:set) 두 함수의 대응관계를 나타낸다. 만약 $x \in X$, $y \in Y$ 일 때 $(x,y) \in f$ 이면 $y=f(x)$라 한다. 
\end{itemize}
\end{definition}

이 때, 다변수함수는 정의역이 여러 집합의 곱집합인 경우 다변수함수라 한다. 예를 들어서, 다음과 같은 함수는 다변수함수이다. 

\begin{example} 
\begin{equation}
(\mathds{R} \times \mathds{R}, \mathds{R}, \{x+y| x \in X, y \in Y \})
\end{equation}
\end{example}


\paragraph{Partial Derivatives} 어떤 다변수함수 $f(x_1, x_2, ... , x_n)$ 에 대해서, $x_i$에 대한 편미분은 다음과 같이 정의된다. 

\begin{definition} 
\begin{equation}
\frac{\partial f}{\partial x_i} = lim_{h->0} \frac{f(x_1, x_2, ..., x_i+h, ... , x_n) - f(x_1, x_2, ..., x_i, ... , x_n) }{h}
\end{equation}
\end{definition}

예를 들어서, $f(x,y) = x^2+y$의 x에 대한 편미분은 $\frac{\partial f}{\partial x} = 2x$ 이다. 

 
\paragraph{Gradient} 
다변수함수 $f(x_1, x_2, ... , x_n)$에 대해서 f의 Gradient는 다음과 같이 정의된다. 
\begin{definition} 
\begin{equation} 
\nabla f =  \sum_i \frac{\partial f}{\partial x_i}
\end{equation}
\end{definition}

\paragraph{Find minimum/maximum} 
다변수함수의 최대/최소, 즉 극값을 가지는 극점(critical point)은 각 변수들에 대한 편미분값이 0이 되는 점이다. 
\paragraph{Coordinate Change and Jacobian} 
$\mathds{R}^n$에서 $\mathds{R}^m$으로 가는 함수 f를 생각하자. 이 때 f의 Jacobian J는 다음과 같이 정의되는 행렬이다. 

\begin{definition} 
\begin{equation}
J_{ij} = \frac{\partial f_i}{\partial x_j}, \qquad i \in [1, m], j \in [1, n]
\end{equation}
\end{definition}

\paragraph{Chain Rule} 

다변수함수 $z = f(x(t), y(t))$에 대해서 
\begin{equation} 
\frac{dz}{dt} = \frac{\partial z}{\partial x} \frac{dx}{dt} + \frac{\partial z}{\partial y}\frac{dy}{dt}
\end{equation}

만약 x,y가 s, t에 대한 다변수함수라면, 

\begin{equation}
\frac{\partial z}{\partial s} = \frac{\partial z}{\partial x}\frac{\partial x}{\partial s} + \frac{\partial z}{\partial y}\frac{\partial y}{\partial s}
\end{equation}

\paragraph{Functionals} 


\subsubsection{Vector Fields}

\paragraph{Definition} 

\paragraph{Curl}

\paragraph{Divergence}



\subsubsection{Line/Surface Integral} 

\paragraph{Introduction} 

\paragraph{Fundamental Theorem of Calculus Revisited}

\paragraph{Green's Theorem} 

\paragraph{Divergence Theorem} 


\subsection{Graph Theory} 



\subsection{Implementation of Network using Python}

\subsection{Stochastic Models} 

\subsubsection{Probability and Random Variable} 

\subsubsection{Markov Chain} 







 
\newpage
\section{Mechanics} 

\subsection{Classical Mechanics} 

\subsection{Thermodynamics} 

\subsection{Quantum Mechanics}  
\newpage
\section{Statistical Mechanics} 

\subsection{Classical Statistical Mechanics}

\subsection{Quantum Statistical Mechanics}  
\newpage
\section{Application on Network Science} 



 

\end{document}  