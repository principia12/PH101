\documentclass[titlepage]{article}

\usepackage{lipsum} % Package to generate dummy text throughout this template

\usepackage[sc]{mathpazo} % Use the Palatino font
\usepackage[T1]{fontenc} % Use 8-bit encoding that has 256 glyphs
\linespread{1.05} % Line spacing - Palatino needs more space between lines
\usepackage{microtype} % Slightly tweak font spacing for aesthetics

\usepackage[hmarginratio=1:1,top=32mm,columnsep=20pt]{geometry} % Document margins
\usepackage{multicol} % Used for the two-column layout of the document
\usepackage[hang, small,labelfont=bf,up,textfont=it,up]{caption} % Custom captions under/above floats in tables or figures
\usepackage{booktabs} % Horizontal rules in tables
\usepackage{float} % Required for tables and figures in the multi-column environment - they need to be placed in specific locations with the [H] (e.g. \begin{table}[H])
\usepackage{hyperref} % For hyperlinks in the PDF

\usepackage{lettrine} % The lettrine is the first enlarged letter at the beginning of the text
\usepackage{paralist} % Used for the compactitem environment which makes bullet points with less space between them

\usepackage{braket}
\usepackage{array}
\usepackage{calc}
\usepackage{graphicx}
\usepackage{listings}
\usepackage{color}
\usepackage[table,xcdraw]{xcolor}
\usepackage{adjustbox}
\usepackage{kotex}
\usepackage{dsfont}

\graphicspath{{images/}}

\definecolor{dkgreen}{rgb}{0,0.6,0}
\definecolor{gray}{rgb}{0.5,0.5,0.5}
\definecolor{mauve}{rgb}{0.58,0,0.82}



\hypersetup{%
    pdfborder = {0 0 0}
}

\newtheorem{definition}{Defnition}[section]
\newtheorem{example}{Example}[section]


\usepackage{abstract} % Allows abstract customization
\renewcommand{\abstractnamefont}{\normalfont\bfseries} % Set the "Abstract" text to bold
\renewcommand{\abstracttextfont}{\normalfont\small\itshape} % Set the abstract itself to small italic text

\usepackage{titlesec} % Allows customization of titles
\renewcommand\thesection{\Roman{section}} % Roman numerals for the sections
\renewcommand\thesubsection{\Roman{subsection}} % Roman numerals for subsections
\titleformat{\section}[block]{\large\scshape\centering}{\thesection.}{1em}{} % Change the look of the section titles
\titleformat{\subsection}[block]{\large}{\thesubsection.}{1em}{} % Change the look of the section titles

\usepackage{fancyhdr} % Headers and footers
\pagestyle{fancy} % All pages have headers and footers
\fancyhead{} % Blank out the default header
\fancyfoot{} % Blank out the default footer
\fancyhead[C]{ PH101 Introductory Physics } % Custom header text
\fancyfoot[RO,LE]{\thepage} % Custom footer text

%----------------------------------------------------------------------------------------
%	TITLE SECTION
%----------------------------------------------------------------------------------------

\title{\vspace{-15mm}\fontsize{24pt}{10pt}\selectfont\textbf{Hitchhiker's Guide to the\\ Introductory Physics}} % Article title

\author{
\large
\textsc{Seungwoo Schin}\\[2mm]
\normalsize School of Computing, KAIST \\ % Your institution
\vspace{-5mm}
}
\date{}

%----------------------------------------------------------------------------------------

\begin{document}

%\maketitle % Insert title

%\thispagestyle{fancy} % All pages have headers and footers

\begin{titlepage}
\maketitle % Insert title

\end{titlepage}


%----------------------------------------------------------------------------------------
%	ARTICLE CONTENTS
%----------------------------------------------------------------------------------------

\thispagestyle{fancy} % All pages have headers and footers

\tableofcontents

\newpage


%----------------------------------------------------------------------------------------
%	ARTICLE CONTENTS
%----------------------------------------------------------------------------------------


\lstset{frame=tb,
  language=Python,
  aboveskip=3mm,
  belowskip=3mm,
  showstringspaces=false,
  columns=flexible,
  basicstyle={\small\ttfamily},
  numbers=none,
  numberstyle=\tiny\color{gray},
  keywordstyle=\color{blue},
  commentstyle=\color{dkgreen},
  stringstyle=\color{mauve},
  breaklines=true,
  breakatwhitespace=true
  tabsize=3
} 


\section{Preface}

본 수업은 네트워크과학에서 종종 응용되는 통계물리학적 기법의 이해를 돕기 위해서 계획되었다. 본격적으로 수업에 들어가기 전에 앞서, 어떠한 수학적 기법들이 어떠한 이유로 역학에 도입되었는지, 또 역학의 궁극적인 목적은 무엇인지에 대해서 살펴보고 왜 통계역학적 기법들이 네트워크과학에서 쓰이는지에 대해서 논해 보고자 한다. 그러한 과정을 통해서 이과에서의 일반적 문제 해결 방법에 대한 대략적인 스케치를 제시하고자 한다. 

\subsection{Introduction to Various Models of Mechanics} 

역학은 주어진 환경에서 물체, 또는 물체들의 움직임을 예측하는 학문이다. 이 정의는 직관적이지만 수학적인 분석에 크게 도움이 되지 않으므로, 조금 더 자세히 각 단어를 살펴보자. 

\begin{itemize} 
\item 주어진 환경 

물리학에서의 환경은 크게 두 가지가 있다. 첫 번째는 물체가 받는 힘이라고 할 수 있다. 모든 힘은 4가지 힘\footnote{만유인력, 전자기력, 약력, 강력}의 조합이다. 힘은 크기와 방향을 가지고 있으며, 힘의 여러 성질들은 벡터로 나타내기에 적합하므로 여기에서는 힘을 벡터, 또는 벡터장으로 기술할 것이다. 두 번째는 물체가 가지는 위치에너지이다. 에너지는 스칼랴로 나타내기 적합하므로, 스칼라장으로 기술된다. 차후 역학 단원에서는 힘과 위치에너지의 관계를 선적분을 이용하여 정의한다. 

\item 움직임을 예측 

 물체의 움직임을 예측한다는 것은 미래의 어떤 시간 t에 대해서 물체의 위치를 특정할 수 있다는 말이다. 이 과정에서 공간 상에서 특정 위치를 수학적으로 다루기 편하게 나타내기 위해서 좌표계의 개념을 도입하고, $\vec{r}(t)$의 변화를 나타내기 위해서 함수의 개념을 사용할 것이다.
\end{itemize}

위 정의에서 볼 때, 역학의 목적은 주어진 벡터장 $\vec{F}(\vec{r})$에 대해서 어떤 물체, 또는 물체들의 위치함수 $\vec{r}(t)$를 구하는 것이다. 이를 구하는 방법에 따라 세 가지 패러다임이 있다. 아래의 세 가지 패러다임은 모두 수학적으로 동치임이므로 주어진 문제의 상황에 따라서 취사선택하여 사용될 수 있다. 

\paragraph{Newtonian Mechanics}

뉴토니안 역학은 뉴턴의 3법칙에 기반하여 역학적 문제를 해결한다. 주어진 환경을 힘으로 기술하는 것이 편리할 때에 주로 사용된다. 수학적으로는 뉴토니안 역학의 문제풀이법은 2계 미분방정식의 해를 구하는 과정으로 생각할 수 있다. 

\paragraph{Lagrangian Mechanics} 

라그랑지안 역학은 액션에 대한 해밀턴의 원리를 이용하여 문제를 해결한다. 액션은 계의 운동에너지에서 위치에너지를 뺀 값으로 정의되며, 이는 다시 계 내의 각 입자의 위치와 그 미분으로 결정되는 범함수(Functional)로 기술할 수 있다. 범함수의 최소값은 변분법에서의 오일러 방정식으로 구할 수 있으므로, 결과적으로 입자의 위치와 그 도함수에 대한 미분방정식을 푸는 것으로 나타난다. 이는 은 주어진 환경이 위치에너지를 이용하여 쉽게 기술될 때 사용된다. 

\paragraph{Hamiltonian Mechanics} 

해밀토니안은 라그랑지안 역학의 일반화된 개념이다. 정규화된 좌표계(canonical coordinate)와 위상 공간(phase space)의 개념을 도입하여 좌표게에 대한 연립미분방정식을 풀어 해를 구한다. 상당히 추상적인 이론이지만, 그 추상성 때문에 여러 분야로 확장이 용이하며, 이는 네트워크 역학의 설명에도 이용될 수 있다\cite{hamiltonian1}. 


\paragraph{Statistical Mechanics} 

위 세 가지의 패러다임은 대개의 역학적 과정을 설명하는 것에 적합하나, 계에 포함된 입자의 수가 많거나 자유도가 높을 경우 해를 구하는 과정이 너무 복잡해진다는 단점이 있다. 단적으로, n-body problem의 시뮬레이션의 시간복잡도는 PSPACE\cite{nbody}임이 밝혀져 있다\footnote{즉, P-NP가 동치가 아닌 것으로 밝혀질 경우 적어도 다항식 시간은 아니며, 사실상 $O(2^n)$ 시간이 걸린다는 뜻이다.} 이에 따라서 통계물리에서는 미시 상태(microstate)와 거시 상태(macrostate)를 정의하고, 미시 상태의 통계적인 양상이 거시 상태에서 관측된다고 가정한다. 이에 따라 확률분포와 앙상블(Ensemble)의 개념을 도입하여 열역학에서 '관측'되었던 거시적인 관측결과를 설명한다. 

\paragraph{Quantum Mechanics} 

양자역학은 고전역학의 패러다임과는 완전히 다른 가정에서 시작한다. \footnote{TBA}


\subsection{Application to Network Science} 

전통적으로 네트워크 분석은 랜덤 그래프에 기반한 Erods-Renyi 모델\cite{erdos}을 이용하여 진행되었다. 하지만 네트워크의 발전 양상이 몇몇 규칙에 강하게 의존하는 것을 기반으로 하여 \cite{barbasi}의 통계역학적 모델이 대두되었다. 이는 통계물리에서 극도로 많은 수의 대상에 대해서 성공적으로 뉴턴 법칙을 적용한 것에 착안하여 제안된 것이라 예상된다. \footnote{본 내용은 정확하지 않을 수 있으며, 앞으로 계속 바뀔 예정입니다.}

다음 단락부터는 위에서 언급된 수학적 내용들을 다루고, 동시에 일반물리 및 고전역학 수준의 고전역학에 대해서 공부해보고자 한다. 

\newpage
\section{Short Recap on Mathematical Tools}

\subsection{Calculus}

\subsubsection{Coordinates}
\label{chap:coordinates}


\paragraph{Various Coordinates} 
좌표계란 공간 내의 임의의 한 점을 어떤 수, 또는 어떤 수들의 순서쌍으로 나타내는 함수이다. 예를 들어서, gps에서 사용하는 위도/경도 등은 지표면상의 위치를 나타내는 하나의 좌표계이다. 본 수업에서 다룰 좌표계에는 크게 4가지의 좌표계가 있다. 

\begin{itemize}
\item{직교좌표계(Cartesian Coordinate)} $(x,y,z)$
\item{극좌표계(Polar Coordinate)} $(r, \theta)$ 점대칭인 2차원 상황에서 많이 이용된다. 
\item{원통 좌표계(Cylindrical Coordinate)} $(r,\theta,z)$원기둥 표면 위의 한 점을 나타내기에 적합하며, 축대칭이면서 동시에 극좌표계를 사용하는 것이 적합할 때 이용된다. 
\item{구면좌표계(Spherical Coordinate)} $(\rho,\theta,\phi)$구면상의 한 점을 나타내기에 적합하다. 보통 점대칭인 상황을 기술할 때 사용하기 용이하다. 물리학에서의 많은 상황은 이러한 점대칭 상황을 가정한다. \footnote{https://youtu.be/StHMKdvuHN0}
\end{itemize}

\paragraph{Transformation between Coordinate Systems} 

\begin{itemize}
\item Cartesian to Polar Coordinate 
\begin{equation}
        r =  \sqrt{x^2+y^2} 
\end{equation}
\begin{equation}
        \theta = tan^{-1} y/x
\end{equation}
\item Cartesian to Cylindrical Coordinate
\item Cartesian to Spherical Coordinate
\end{itemize}



\subsubsection{Vector} 

\paragraph{Definition} 벡터의 정의는 쓰고자 하는 범위에 따라서 여러가지로 달라지나, 여기서는 가장 naive한 정의인 크기와 방향을 가지는 객체라고 정의\footnote{벡터의 정의는 다양하며, 여기서 말하는 크기와 방향이라는 개념 또한 수학적으로 애매하지만 일단은 넘어가고 차후에 더 수학적으로 엄밀한 정의를 할 수 있을 것이다.} 하자. 이와 대응되는 개념의 스칼라는 크기만 가진 개념으로 정의하자. 

예를 들어서 온도나 길이 등은 크기만 가지는 개념으로, 스칼라양으로 볼 수 있다. 이와는 다르게, 힘 같은 것은 크기와 방향을 동시에 가지므로 벡터로 생각할 수 있다. 

\paragraph{Operations} 

\begin{itemize}
\item Addition, Subtraction 
\item Scalar Multiplication 
\item Inner Product
\item Outer Product (3차원에서만 적용가능)
\end{itemize}


\subsubsection{Multivariable Functions} 

\paragraph{Recap on Definition of Function}  함수는 다음과 같은 3-tuple로 정의된다. 
\begin{definition} 
함수 f는 (X, Y, f)로 정의되며, 
\begin{itemize} 
\item X (range:set) 정의역이라고도 하며, 함수가 받는 입력의 집합이다. 
\item Y (range:set) 공역이라고 하며, 함수가 반환하는 출력의 집합이다. 
\item $f \subset X \times Y$ (range:set) 두 함수의 대응관계를 나타낸다. 만약 $x \in X$, $y \in Y$ 일 때 $(x,y) \in f$ 이면 $y=f(x)$라 한다. 
\end{itemize}
\end{definition}

이 때, 다변수함수는 정의역이 여러 집합의 곱집합인 경우 다변수함수라 한다. 예를 들어서, 다음과 같은 함수는 다변수함수이다. 

\begin{example} 
\begin{equation}
(\mathds{R} \times \mathds{R}, \mathds{R}, \{x+y| x \in X, y \in Y \})
\end{equation}
\end{example}


\paragraph{Partial Derivatives} 어떤 다변수함수 $f(x_1, x_2, ... , x_n)$ 에 대해서, $x_i$에 대한 편미분은 다음과 같이 정의된다. 

\begin{definition} 
\begin{equation}
\frac{\partial f}{\partial x_i} = lim_{h->0} \frac{f(x_1, x_2, ..., x_i+h, ... , x_n) - f(x_1, x_2, ..., x_i, ... , x_n) }{h}
\end{equation}
\end{definition}

예를 들어서, $f(x,y) = x^2+y$의 x에 대한 편미분은 $\frac{\partial f}{\partial x} = 2x$ 이다. 

 
\paragraph{Gradient} 
다변수함수 $f(x_1, x_2, ... , x_n)$에 대해서 f의 Gradient는 다음과 같이 정의된다. 
\begin{definition} 
\begin{equation} 
\nabla f =  \sum_i \frac{\partial f}{\partial x_i}
\end{equation}
\end{definition}

\paragraph{Find minimum/maximum} 
다변수함수의 최대/최소, 즉 극값을 가지는 극점(critical point)은 각 변수들에 대한 편미분값이 0이 되는 점이다. 
\paragraph{Coordinate Change and Jacobian} 
$\mathds{R}^n$에서 $\mathds{R}^m$으로 가는 함수 f를 생각하자. 이 때 f의 Jacobian J는 다음과 같이 정의되는 행렬이다. 

\begin{definition} 
\begin{equation}
J_{ij} = \frac{\partial f_i}{\partial x_j}, \qquad i \in [1, m], j \in [1, n]
\end{equation}
\end{definition}

\paragraph{Chain Rule} 

다변수함수 $z = f(x(t), y(t))$에 대해서 
\begin{equation} 
\frac{dz}{dt} = \frac{\partial z}{\partial x} \frac{dx}{dt} + \frac{\partial z}{\partial y}\frac{dy}{dt}
\end{equation}

만약 x,y가 s, t에 대한 다변수함수라면, 

\begin{equation}
\frac{\partial z}{\partial s} = \frac{\partial z}{\partial x}\frac{\partial x}{\partial s} + \frac{\partial z}{\partial y}\frac{\partial y}{\partial s}
\end{equation}

\paragraph{Functionals} 


\subsubsection{Vector Fields}

\paragraph{Definition} 

\paragraph{Curl}

\paragraph{Divergence}



\subsubsection{Line/Surface Integral} 

\paragraph{Introduction} 

\paragraph{Fundamental Theorem of Calculus Revisited}

\paragraph{Green's Theorem} 

\paragraph{Divergence Theorem} 


\subsection{Graph Theory} 



\subsection{Implementation of Network using Python}

\subsection{Stochastic Models} 

\subsubsection{Probability and Random Variable} 

\subsubsection{Markov Chain} 







 
\newpage
\section{Mechanics} 

\subsection{Classical Mechanics} 

\subsection{Thermodynamics} 

\subsection{Quantum Mechanics}  
\newpage
\section{Statistical Mechanics} 

\subsection{Classical Statistical Mechanics}

\subsection{Quantum Statistical Mechanics}  
\newpage
\section{Application on Network Science} 



 

\newpage
\bibliographystyle{ieeetr}
\bibliography{reference} 



\end{document}  